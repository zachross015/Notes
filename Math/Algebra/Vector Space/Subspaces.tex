\section{Subspaces}\label{subspaces}

Within a vector space, we have the possibility of the vector space containting a
\emph{subspace}, which is a vector space that is a subset of another vector
space. A subscpace is notated using the same notation as a subset. Given a
vector space $V$ and a subscpace $U$, we represent this as such.
\begin{equation*}
    U \subseteq V
\end{equation*}

Proving that a subset is a subspace is almost the same as proving that it is a
vector space to begin with. We must establish that the vector space follows the
axioms defined in section \ref{generalvectorspaces}, and that their results
exist in their own vector space as well as the superspace.

\begin{exmp}
    Given a three dimensional vector space $V$ and a two dimensional vector
    space $W$, we want to prove that $W$ is a subspace of $V$.
    \begin{align*}
        &V = \mathbb{R}^3 = \{(x,y,z): x,y,z \in \mathbb{R}\}\\
        &W = \mathbb{R}^2 = \{(x,y): x,y \in \mathbb{R}\}
    \end{align*}
    This example has a fairly simple solution since we can think of $W$ as the
    three dimensional vector space using a constant $0$ as the third dimension.
    In other words, $W$ will be the $x$ and $y$ planes in $\mathbb{R}^3$.
    \begin{align*}
        &V = \mathbb{R}^3 = \{(x,y,z): x,y,z \in \mathbb{R}\}\\
        &W = \mathbb{R}^3 = \{(x,y,0): x,y \in \mathbb{R}\}
    \end{align*}
    Because we know that $\mathbb{R}^3$ is vector space already, this concludes
    the proof of the equation. 
    \begin{equation*}
        W \subseteq V
    \end{equation*}
\end{exmp}

By the logic used in this example, we can also show that each dimension of real
numbers is a subspace of one higher dimension. E.g.
\begin{equation*}
    \mathbb{R}^1 \subseteq \mathbb{R}^2 \subseteq \hdots \subseteq
    \mathbb{R}^\infty
\end{equation*}

\begin{exmp}
    Given $V$ a vector space of all continuous functions on $(-\infty ,
    \infty)$ and a vector space of all polynomials $P$, prove that $P
    \subseteq V$.

    This solution is a little trickier than the last, but still simple enough.
    Say we have two polynomial functions $p, q$ and some constant $k$, where the
    polynomial functions are
    \begin{align*}
        p(x) = a_0 + a_1x +a_2x^2 + \hdots + a_nx^n\\
        q(x) = b_0 + b_1x +b_2x^2 + \hdots + b_nx^n
    \end{align*}
    To prove the continuity, we first show that $p, q$ are additive. Because the
    addition of two polynomial functions is still in the vector space of all
    polynomial functions, we are only evaluating whether or not a polynomial can
    exist in the domain of all continuous functions. The result of multiplying a
    polynomial by a scalar has the same effect, and therefore the result is
    within the domain of polynomials. Now, we are just proving that a polynomial
    exists in the domain of all continuous functions.
    \begin{align*}
        \lim_{x \to m} f(x) = \lim_{x \to m} (a_0 + a_1x +a_2x^2 + \hdots +
        a_nx^n)\\
        lim_{x \to m } a_0 
        + lim_{x \to m } a_1x 
        + lim_{x \to m } a_2x^2 
        + \hdots 
        + lim_{x \to m } a_nx^n
    \end{align*}
    This is of the form $a_0 + m^n$. Because the value of $n$ never goes below
    $0$, this function is always continuous.
\end{exmp}
