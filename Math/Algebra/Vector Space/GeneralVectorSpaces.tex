\section{General Vector Spaces}\label{generalvectorspaces}
Any vector exists in something called a \emph{vector space}, which is any set on
which there are defined two operations
\begin{enumerate}
    \item Addition: If two vectors, $u$ and $v$, both exist in a vector space
        $V$, then the addition of $u$ and $v$ must also exist in $V$.    
    \item Scalar multiplication: If some vector $u$ exists in a vector $V$,
        and some scalar $k$ exists in the domain of all real numbers, then $k$
        times $u$ must also exist in $V$.
\end{enumerate}
As an extension of this definition, we have a system of operations, or axiums,
that must be satisfied in order for the set of vectors to be considered a vector
space. 
\begin{enumerate}
    \item The addition property stated previously must hold true.
    \item The addition of two vectors in a vector space must be commutative.
        e.g. $u + v = v + u$
    \item The addition of two vectors in a vector space must be associative.
        e.g. $(u + v) + w = u + (v + w)$
    \item There must exist a zero vector in the vector space. e.g. $0 + u = u +
        0 = u$
    \item For every vector in the vector space, there must exist an equal
        and opposite vector. e.g. $u + (-u) = (-u) + u = 0$
    \item The multiplication property stated previously must hold true.
    \item The multiplication of a scalar with two added vectors must be
        ditributive. e.g. $k(u + v) = ku + kv$ 
    \item The multiplication of a vector with two added scalars must be
        ditributive. e.g. $(k + l)u = ku + lu$ 
    \item The multiplication of two scalars and a vector must be associative.
        e.g. $(kl)u = k(lu)$
    \item There exists an identity vector, such that multiplying a vector by
        this identity will leave the vector unchanged. e.g. $1 * u = u$
\end{enumerate}
