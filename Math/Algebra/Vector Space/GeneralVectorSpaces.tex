\section{General Vector Spaces}\label{generalvectorspaces}
Any vector exists in something called a \emph{vector space}, which is any set on
which there are defined two operations
\begin{enumerate}
    \item Addition: If two vectors, $u$ and $v$, both exist in a vector space
        $V$, then the addition of $u$ and $v$ must also exist in $V$.    
    \item Scalar multiplication: If some vector $u$ exists in a vector $V$,
        and some scalar $k$ exists in the domain of all real numbers, then $k$
        times $u$ must also exist in $V$.
\end{enumerate}
As an extension of this definition, we have a system of operations, or axiums,
that must be satisfied in order for the set of vectors to be considered a vector
space. 
\begin{enumerate}
    \item The addition property stated previously must hold true.
    \item The addition of two vectors in a vector space must be commutative.
        e.g. $u + v = v + u$
    \item The addition of two vectors in a vector space must be associative.
        e.g. $(u + v) + w = u + (v + w)$
    \item There must exist a zero vector in the vector space. e.g. $0 + u = u +
        0 = u$
    \item For every vector in the vector space, there must exist an equal
        and opposite vector. e.g. $u + (-u) = (-u) + u = 0$
    \item The multiplication property stated previously must hold true.
    \item The multiplication of a scalar with two added vectors must be
        ditributive. e.g. $k(u + v) = ku + kv$ 
    \item The multiplication of a vector with two added scalars must be
        ditributive. e.g. $(k + l)u = ku + lu$ 
    \item The multiplication of two scalars and a vector must be associative.
        e.g. $(kl)u = k(lu)$
    \item There exists an identity vector, such that multiplying a vector by
        this identity will leave the vector unchanged. e.g. $1 * u = u$
\end{enumerate}

\begin{exmp}
    We want to prove that the two dimensional vector space $V$ is a valid vector
    space.  First off, we should establish what are the elements of the vector
    we are evaluating.
    \begin{equation*}
        V = \mathbb{R}^2 = \{(x,y):x,y \in \mathbb{R} \}
    \end{equation*}
    Now this equation may be a little bit confusing, so let's take a look at
    each part of it.

    We are dealing with the vector space $V$, which is a collection of vectors
    that exist in two dimensions. The notation of $\mathbb{R}$, or the domain of
    all real numbers, is in itself a dimension that a vector can exist in. Since
    this vector space is two dimensional, we show that by saying this is in
    $\mathbb{R}*\mathbb{R}$ or $\mathbb{R}^2$. 

    The final part serves as a rule for what can exist in the
    vector space. A verbose version of this is that $x$ and $y$ must satisfy the
    constraint of $(x,y)$ for all $x$ and $y$ in the domain of all real numbers.
    In this situation, this is not much of a constraint, as it can be satisfied
    by all real numbers.

    We are then left with proving this, which comes quite simply. We take two
    independent and abstract vectors $u$ and $v$ in $V$, with values $(a,b)$ and
    $(c,d)$ respectively. By addition, we have
    \begin{equation*}
        u + v = (a + c, b + d) \in V
    \end{equation*}
    Since the addition of $a$ and $c$, and $b$ and $d$, all still exist in
    $\mathbb{R}$, the addition property holds true.

    For multiplication, we use the abstract vector $u$ from before, and an
    abstract scalar $k$ that exists in $\mathbb{R}$. Then
    \begin{equation*}
        ku = (ka, kb) \in V
    \end{equation*}

    The rest of the axiums can be easily proven after these two arguments have
    been satisfied.
\end{exmp}

