\section{Gaussian Elimination} \label{gaussianelimination}
\emph{Guassian elimination} is a method of reducing a matrix to a form called the
row-echlon form. The intention is to minimize the amount of variables before
finding the solution to a system of equations. It uses a concept called \emph{Leading
Ones}, which means the first non-zero entry is a one. The reduced row-echelon
matrix is defined by a series of requisites:
\begin{enumerate}
    \item The first non-zero in any row is a one
    \item If a row in the matrix contains all zeroes, that row must be on the
        bottom.
    \item Leading Ones move to the right. 
    \item Above and below each leading one is a zero.
\end{enumerate}
The form is only considered reduced if the fourth condition is met.

In order to find the reduced row-echlon matrix, there are a few elementary
operations that can be performed on the matrix.
\begin{enumerate}
    \item Interchange two rows
    \item Multiply a row by a non-zero constant
    \item Add a multple of one row to another row
\end{enumerate}
Operations performed on the matrix affect all elements of the row, including the
constant in the case that an augmented matrix is being used.  Once we have the
reduced row-echelon matrix as an augmented matrix, we can convert this into a
system of equations where we solve for each variable. Using a method called back
substitution makes it much simpler. Back substitution is simply solving for
leading variables by pluggin already identified values into
each system.

\begin{exmp}
    Given the following matrix:
    \begin{equation*}
        \begin{bmatrix}[cccccc|c]
            1 & 3 & -2 &  0 & 2 &  0 &  0 \\
            2 & 6 & -5 & -2 & 4 & -3 & -1 \\
            0 & 0 &  5 & 10 & 4 & 15 &  5 \\
            2 & 6 &  0 &  8 & 4 & 18 &  6 \\
        \end{bmatrix}
    \end{equation*}
    we want to try and find the solution to the system of equations that it
    corresponds to. After implementing a few of the row reduction
    operations, we end up with a matrix that looks more like this:
    \begin{equation*}
        \begin{bmatrix}[cccccc|c]
            1 & 3 & -2 &  0 & 2 &  0 &  0           \\
            0 & 0 &  1 &  2 & 0 &  3 &  1           \\
            0 & 0 &  0 &  0 & 0 &  1 &  \frac{1}{3} \\
            0 & 0 &  0 &  0 & 0 &  0 &  0           \\
        \end{bmatrix}
    \end{equation*}
    With this new matrix, we can separate the variables into a set of equations. 
    \begin{align*}
        &x_1 &+ &3x_2 &- &2x_3 &+ &0    &+ &2x_5 &+ &0x_6 &= 0 \\
        &0   &+ &0    &+ &x_3  &+ &2x_4 &+ &0    &+ &3x_6 &= 1 \\
        &0   &+ &0    &+ &0    &+ &0    &+ &0    &+ &x_6  &= \frac{1}{3} \\
        &0   &+ &0    &+ &0    &+ &0    &+ &0    &+ &0    &= 0 
    \end{align*}
    Removing the leading zeros, we get:
    \begin{align*}
        x_1 + 3x_2 - 2x_3 + 2x_5 = 0 \\
        x_3 + 2x_4 + 3x_6 = 1 \\
        x_6 = \frac{1}{3} \\
    \end{align*}
    From here, we want to try and reduce the set to as many unknowns as
    possible. We begin by solving each equation for the least significant number
    in the corresponding equation.
    \begin{align*}
        -3x_2 + 2x_3 - 2x_5 = x_1 \\
        1 - 2x_4 - 3x_6 = x_3 \\
        \frac{1}{3} = x_6 \\
    \end{align*}
    Now we can use back substitution to solve for the rest of the system, and
    identify the unknowns.
    \begin{align*}
        x_1 = -3x_2 + -4x_4 - 2x_5 \\
    \end{align*}
    After solving, we are left with the non-leading variables $x_2$, $x_4$, and
    $x_5$. These are the unkowns that our solution relies on. From this, we
    identify the domains of the unknowns.
    \begin{equation*}
        x_2 = r, x_4 = s, x_5 = t \text{ for } r,s,t \in \mathbb{R}
    \end{equation*}
    Then plug it all in for the solution.
    \begin{equation*}
        x_1 = -3r - 4s - 2t
    \end{equation*}
\end{exmp}

