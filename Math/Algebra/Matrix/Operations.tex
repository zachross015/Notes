\section{Matrix Operations}

We will now define an priliminary set of operations that can be performed on a
matrix of any size.

Given two matrices, $A$ and $B$ of arbitrary sizes, we say that matrix $A$ and
$B$ are equivalent if and only if 
\begin{enumerate}
    \item Their row counts and column counts are equal
    \item $a_{ij} = b_{ij}$, for all $i$ in $m$ and all $j$ in $n$
\end{enumerate}

The addition of matrix $A$ and $B$ can only be done if their row counts and
column counts are equal. It is performed as follows:
\begin{equation}
    A+B = 
    \begin{bmatrix}
        a_{11} + b_{11} & a_{12} + b_{12} & \ldots & a_{1n} + b_{1n} \\
        a_{21} + b_{21} & a_{22} + b_{22} & \ldots & a_{2n} + b_{2n} \\
        \vdots & \vdots & \ddots & \vdots \\
        a_{m1} + b_{m1} & a_{m2} + b_{m2} & \ldots & a_{mn} + b_{mn} \\
    \end{bmatrix}
\end{equation}

The scalar multiplication of a matrix $A$ and a scalar $c$ is defined as
\begin{equation}
    cA = 
    \begin{bmatrix}
        c*a_{11} & c*a_{12} & \ldots & c*a_{1n} \\
        c*a_{21} & c*a_{22} & \ldots & c*a_{2n} \\
        \vdots & \vdots & \ddots & \vdots \\
        c*a_{m1} & c*a_{m2} & \ldots & c*a_{mn} \\
    \end{bmatrix}
\end{equation}

The multiplication of two matrices $A$ and $B$ can only be fulfilled under the
condition that the column count of the initial matrix is equal to the row count
of the multiplicant(e.g. $n_1 = m_2$). The resulting matrix will contain the row
count of the initial matrix with the column count of the multiplicant. Each
element of the matrix is determined by the summation of the corresponding row in
the initial matrix, multiplied by the summation of the corresponding column in
the multiplicant matrix.
\begin{equation}
    AB = 
    \begin{bmatrix}
        \sum_{i=1}^{n}a_{1i}\sum_{j=1}^{m}b_{j1} 
        & \sum_{i=1}^{n}a_{1i}\sum_{j=1}^{m}b_{j2} 
        & \ldots 
        & \sum_{i=1}^{n}a_{1i}\sum_{j=1}^{m}b_{jn} \\
        \sum_{i=1}^{n}a_{2i}\sum_{j=1}^{m}b_{j1} 
        & \sum_{i=1}^{n}a_{2i}\sum_{j=1}^{m}b_{j2} 
        & \ldots 
        & \sum_{i=1}^{n}a_{2i}\sum_{j=1}^{m}b_{jn} \\
        \vdots & \vdots & \ddots & \vdots \\
        \sum_{i=1}^{n}a_{mi}\sum_{j=1}^{m}b_{j1} 
        & \sum_{i=1}^{n}a_{mi}\sum_{j=1}^{m}b_{j2} 
        & \ldots 
        & \sum_{i=1}^{n}a_{mi}\sum_{j=1}^{m}b_{jn} \\
    \end{bmatrix}
\end{equation}
This may be a little confusing to remember at a glance, so remember that its the
sum of each row in the first matrix times the sum of each column in the second
one. It is important to note that $AB \neq BA$. This is easily proved since the 
column count of one matrix may not equal the row count of the other, even when
the converse is equivalent.

From any of these matrices, we can derive a set of properties that each matrix
contains.
\begin{enumerate}
    \item $A+B=B+A$ [commutative]
        \begin{proof}
            Let $A$ and $B$ be matrices of arbitrary, but equivalent, sizes. The
            addition of the two matrices is commutative because the addition of
            each individual element of both matrices is commutative.
        \end{proof}
    \item $(A+B)+C=A+(B+C)$ [associative]
        \begin{proof}
            The associativity of each matrix is proved by the associativity of
            its elements.
        \end{proof}
    \item $(AB)C=A(BC)$ [associative]
        % TODO: Write a proof for this
    \item $A(B+C)=AB+AC$ [distributive]
        % TODO: Write a proof for this
    \item $a(B+C)=aB+aC$ [scalar distributive]
        % TODO: Write a proof for this
    \item $(a+b)C=aC+bC$ [scalar distributive]
        \begin{proof}
            \begin{align*}
                (a+b)C &=
                \begin{bmatrix}
                    (a+b)*c_{11} & (a+b)*c_{12} & \ldots & (a+b)*c_{1n} \\
                    (a+b)*c_{21} & (a+b)*c_{22} & \ldots & (a+b)*c_{2n} \\
                    \vdots & \vdots & \ddots & \vdots \\
                    (a+b)*c_{m1} & (a+b)*c_{m2} & \ldots & (a+b)*c_{mn} \\
                \end{bmatrix} \\
                &=
                \begin{bmatrix}
                    a*c_{11}+b*c_{11}
                    & a*c_{12}+b*c_{12}
                    & \ldots
                    & a*c_{1n}+b*c_{1n} \\
                    a*c_{21}+b*c_{21}
                    & a*c_{22}+b*c_{22}
                    & \ldots
                    & a*c_{2n}+b*c_{2n} \\
                    \vdots & \vdots & \ddots & \vdots \\
                    a*c_{mn}+b*c_{mn}
                    & a*c_{m1}+b*c_{m1} 
                    & a*c_{m2}+b*c_{m2} 
                    & \ldots 
                    a*c_{mn}+b*c_{mn} \\
                \end{bmatrix}\\
                &= aC + bC
            \end{align*}
        \end{proof}
    \item $(ab)C=a(bC)$ [scalar associative]
        % TODO: Write a proof for this 
    \item $(aB)C=a(BC)$ [scalar associative]
        % TODO: Write a proof for this
\end{enumerate}

For any system of equations, we have the following theorem:
\begin{theorem}\label{lesolutions}
    Every system of linear equations either has
    \begin{enumerate}
        \item Infinite solutions,
        \item Zero solutions,
        \item or Exactly one solution
    \end{enumerate}
\end{theorem}

\begin{proof}
    Let $AX=B$ be a linear system. If it has no solutions, then the only
    possibilities for a solution left are $1$ and $3$. Suppose the system has
    two solutions, $X_1$ and $X_2$. Thus $AX_1=AX_2=\ldots=AX_n=B$.

    Let $X_0=X_1-X_2\neq 0$. We can claim that $X_1+kX_0$ is a solution for all
    $k \in \mathbb{R}$
    \begin{align*}
        A(X_1+kX_0)
        &= AX_1+kAX_0 \\
        &= B+kAX_0 \\
        &= B+kA(X_1-X_2) \\
        &= B+k(AX_1-AX_2) \\
        &= B+k(B-B)\\
        &= B+k(0)\\
        &= B
    \end{align*}
\end{proof}

Take for example the system of linear equations $x + y = 1$ and $2x + 2y =
2$. Solving for $y$ in either case gives us the equation $y=-x + 1$, giving
us an infinite set of solutions for this system of equations because the
lines run parallel and coincide with one another. Another example would be
the system of equations $y=x+1$ and $y=x+3$. Given a cursory look, it is
obvious that there is no solution because the lines run parallel and never
intersect.
