\section{Homogeneous Linear Systems}
A \emph{homogeneous linear system} is a system of linear equations for which all
constant terms are zero. The property of having all constant terms equal to zero
is called having the \emph{trivial solution}.  Such a system takes the form 
\begin{equation}\label{homogeneousls}
    \begin{split}
        a_{11}x_1 + a_{12}x_2+&\ldots+a_{1n}x_n  = 0\\
        a_{21}x_1 + a_{22}x_2+&\ldots+a_{2n}x_n  = 0\\
                              &\vdots\\
        a_{m1}x_1 + a_{m2}x_2+&\ldots+a_{mn}x_n  = 0\\
    \end{split}
\end{equation}
This is equivalent to a matrix of the form 
\begin{equation}
    Ax = 0
\end{equation}
Where $A$ is an $m*n$ matrix, $x$ is a column vector with $n$ entries, and $0$
is the zero vector with $m$ entries. Taking a closer look at the innate
properties of a  homogeneous linear system and using Theorem \ref{lesolutions},
it is apparent that there can only ever be one solution or infinite solutions.
In other words, it makes the solution \emph{consistent}. It can not have zero
solutions, since there will always be exactly as many solutions as unknowns, or
more unknowns than solutions. In the event that the amount of solutions and
unknowns are the same, it is easy to row reduce the elements into an identity
matrix, which will result in exactly one solution for each unknown . When there
are more unknowns than solutions, we have a set of uknowns that exist in the
domain of all real numbers, so there are infinite possiblites for a solution.
Putting it all together, we get the theorem
\begin{theorem}\label{infsolutions}
    A homogeneous linear system with more unknowns than equations has infinitely
    many solutions.
\end{theorem}
% TODO: Write a proof for this
The proof for this is beyond my current understanding of the subject, so for now
it is left as an excercise for the reader.
